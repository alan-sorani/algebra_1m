\RequirePackage{luatex85}
\PassOptionsToPackage{luatex}{hyperref}

\documentclass[article, 10pt,oneside]{article}

%\counterwithout{section}{chapter}

%%%%%%%%%%%%%
%  Geometry %
%%%%%%%%%%%%%%

\RequirePackage{geometry}
\geometry{
	 a4paper,
	 inner=20mm,
	 outer=30mm,
	 marginparwidth = 20mm,
	 top = 20mm,
	 bottom = 20mm
	}

%%%%%%%%%
% Babel %
%%%%%%%%%

\usepackage[nil,bidi=basic-r]{babel}
\babelprovide[import=he,main]{hebrew}
\babelprovide[import=en-GB]{english}

% For some reason Babel’s `\babelfont` doesn’t work
\babelfont[hebrew]{rm}{Open Sans Hebrew}

\newcommand{\texthebrew}[1]{\foreignlanguage{hebrew}{#1}}
\newcommand{\nikud}[1]{$\mbox{\H{#1}}$}
\newcommand{\textenglish}[1]{\foreignlanguage{english}{#1}}
\newcommand{\LR}[1]{{‏\textdir TLT #1}}

%%%%%%%%%%%%
% Styling %
%%%%%%%%%%%%

\renewcommand{\emph}[1]{\textbf{#1}}

%%%%%%%%%%
% Maths  %
%%%%%%%%%%

\usepackage{math-symbols}
\usepackage{math-fonts}
\usepackage{math-graphics}
\usepackage{math-theorems-heb}

%long division

\usepackage{polynom}
\makeatletter
\def\pld@CF@loop#1+{%
    \ifx\relax#1\else
        \begingroup
          \pld@AccuSetX11%
          \def\pld@frac{{}{}}\let\pld@symbols\@empty\let\pld@vars\@empty
          \pld@false
          #1%
          \let\pld@temp\@empty
          \pld@AccuIfOne{}{\pld@AccuGet\pld@temp
                            \edef\pld@temp{\noexpand\pld@R\pld@temp}}%
           \pld@if \pld@Extend\pld@temp{\expandafter\pld@F\pld@frac}\fi
           \expandafter\pld@CF@loop@\pld@symbols\relax\@empty
           \expandafter\pld@CF@loop@\pld@vars\relax\@empty
           \ifx\@empty\pld@temp
               \def\pld@temp{\pld@R11}%
           \fi
          \global\let\@gtempa\pld@temp
        \endgroup
        \ifx\@empty\@gtempa\else
            \pld@ExtendPoly\pld@tempoly\@gtempa
        \fi
        \expandafter\pld@CF@loop
    \fi}
\def\pld@CMAddToTempoly{%
    \pld@AccuGet\pld@temp\edef\pld@temp{\noexpand\pld@R\pld@temp}%
    \pld@CondenseMonomials\pld@false\pld@symbols
    \ifx\pld@symbols\@empty \else
        \pld@ExtendPoly\pld@temp\pld@symbols
    \fi
    \ifx\pld@temp\@empty \else
        \pld@if
            \expandafter\pld@IfSum\expandafter{\pld@temp}%
                {\expandafter\def\expandafter\pld@temp\expandafter
                    {\expandafter\pld@F\expandafter{\pld@temp}{}}}%
                {}%
        \fi
        \pld@ExtendPoly\pld@tempoly\pld@temp
        \pld@Extend\pld@tempoly{\pld@monom}%
    \fi}
\makeatother

%%%%%%%%%%%%%
% Counters  %
%%%%%%%%%%%%%

\setcounter{section}{1}

%%%%%%%%%%
% Title  %
%%%%%%%%%%
\title{
אלגברה 1מ' \\ תרגול 1 -- פולינומים
}
\author{אלעד צורני}
\date{
25 באוקטובר 2020
}

\begin{document}
\maketitle

\subsection*{הגדרות בסיסיות}

\begin{definition}[פולינום]
\emph{פולינום במשתנה x מעל שדה $\mathbb{F}$}
הוא ביטוי מהצורה
\[
f\prs{x} = a_n x^n + a_{n-1} x^{n-1} \ldots + a_1 x + a_0
\]
כאשר
$a_0, \ldots, a_n \in \mathbb{F}$.\\
לפעמים נכתוב זאת בקצרה
\[\text{.} f\prs{x} = \sum_{i = 0}^n a_i x^i\]
\end{definition}

\begin{examples*}[פולינומים]
\enumthm
\begin{enumerate}
\item $0$
\item $42$
\item $x-16$
\item $x^2 - 7x - 2$
\item $3 x^2$ 
\end{enumerate}
\end{examples*}

\begin{definition}[הדרגה של פולינום]
עבור מספר אי־שלילי
$m$
נאמר ש%
\emph{הדרגה של פולינום
$f(x)$
היא
$m$}
ונסמן
\[\deg \prs{f} \coloneqq m\]
אם יש איברים
$a_0, \ldots, a_n \in \mathbb{F}$
עבורם
$a_n \neq 0$
וגם
\[\text{.} f(x) = a_n x^n + a_{m-1} x^{m-1} + \ldots + a_1 x + a_0\]
עבור
$f(x) = 0$
נגדיר את הדרגה של
$f$
להיות
$\deg\prs{f} \coloneqq -\infty$.
\end{definition}

\begin{definition}[מקדם מוביל]
עבור פולינום
$f(x)$
מדרגה
$n$
המקדם
$a_n$
נקרא
\emph{המקדם המוביל של
$f$}.
\end{definition}

\begin{definition}[פולינום מוֹני]
אם המקדם המוביל של פולינום
$f(x)$
שווה
$1$,
נאמר ש־%
\emph{$f$
פולינום מוֹני%
}.
\end{definition}

\begin{definition}[מקדם מוביל]
עבור פולינום
$f(x)$,
המקדם
$a_0$
נקרא
\emph{המקדם המוביל של
$f$}
(גם אם
$a_0 = 0$%
).
\end{definition}

\begin{notation}[מרחב פולינומים]
אוסף הפולינומים במשתנה
$x$
מעל שדה
$\mathbb{F}$
מסומן
$\mathbb{F}\brs{x}$.
\end{notation}

\begin{notation}[פולינומים עד דרגה
$n$]
מרחב הפולינומים במשתנה
$x$
מעל שדה
$\mathbb{F}$
שדרגתם לכל היותר
$n$
מסומן
$\mathbb{F}_n\brs{x}$.
\end{notation}

\subsection*{חיבור, כפל וחילוק פולינומים}

כדי לחבר פולינומים, נחבר את המקדמים שלהם.

\begin{definition}[חיבור פולינומים]
בהינתן שני פולינומים
\[f\prs{x} = \sum_{i = 0}^n a_i x^i, \quad g\prs{x} = \sum_{j=0}^m b_i x^i\]
נגדיר את
\emph{הסכום שלהם}
על ידי
\[\prs{f+g}\prs{x} \coloneqq \sum_{i=0}^{\max\set{n,m}} \prs{a_i + b_i} x^i\]
עם הסימון
$a_i = 0$
עבור
$i > n$
והסימון
$b_j = 0$
עבור
$j > m$.
\end{definition}

\begin{remark}
מתקיים
\[\text{.}\deg\prs{f+g} \leq \max\prs{\deg\prs{f}, \deg\prs{g}}\]
\end{remark}

\begin{example}
\[\prs{x^3 + 17x - 42} + \prs{-x^3 + x^2 - 17x + 3} = \prs{1-1}x^3 + 1 \cdot x^2 + \prs{17-17}x + \prs{-42 + 3} = x^2 - 39\]
\end{example}

כדי לכפול פולינומים, נחבר את מכפלות הגורמים שלהם.

\begin{definition}[כפל פולינומים]
עבור פולינומים
\[f\prs{x} = \sum_{i = 0}^n a_i x^i, \quad g\prs{x} = \sum_{j = 0}^m b_i x^i\]
נגדיר את
\emph{המכפלה שלהם}
על ידי
\[\text{.} \prs{f \cdot g}\prs{x} \coloneqq \sum_{d = 0}^{n+m} \prs{\sum_{i = 0}^d a_i b_{d-i}} x^d\]
בדרך כלל נסמן
$fg$
במקום
$f \cdot g$.
\end{definition}

\begin{remark}
מתקיים
\[\text{.} \deg\prs{f \cdot g} \leq \deg\prs{f} + \deg\prs{g}\]
\end{remark}

\begin{example}
\begin{align*}
\prs{x^3 + 6x - 5} \prs{2 x^2 + 3 x - 7} &=
\prs{\prs{-5} \cdot \prs{-7}} + \prs{-5 \cdot 3 + 6 \cdot \prs{-7}} x + \prs{-5 \cdot 2 + 6 \cdot 3} x^2 \\&\phantom{=} + \prs{6 \cdot 2 + 1 \cdot \prs{-7}} x^3 + \prs{1 \cdot 3} x^4 + \prs{1 \cdot 2} x^5
\\&=
35 + \prs{-15-42} x + \prs{-10 + 18} x^2 + \prs{12 - 7} x^3 + 3 x^4 + 2 x^5
\\&=
2x^5 + 3x^4 + 5 x^3 + 8 x^2 - 57 x + 35 
\end{align*}
\end{example}

\begin{definition}[חילוק פולינומים]
עבור פולינומים
$f,g,r \in \mathbb{F}\brs{x}$
נגיד כי
\emph{
$f$
מתחלק ב־%
$g$
עם שארית
$r$
}
אם קיים פולינום
$h \in \mathbb{F}\brs{x}$
עבורו
$f = hg + r$
וגם
$\deg\prs{r} \leq \deg\prs{g}$.
\\
אם קיים פולינום
$h \in \mathbb{F}\brs{x}$
עבורו
$f = hg$
נאמר כי
\emph{
$f$
מתחלק ב־%
$g$
},
נגיד ש־%
$h$
שווה לחלוקה של
$f$
ב־%
$g$
ונסמן
\[\text{.} h = \frac{f}{g}\]
במקרה זה גם נאמר ש־%
$g$
מחלק את
$f$
ונסמן
$g \mid f$.
\end{definition}

\begin{algorithm}[חילוק ארוך של פולינומים]
נתאר אלגוריתם לחילוק פולינום
$f$
בפולינום
$g$
עם שארית.

נציג תוך כדי זאת דוגמה בה
\[\text{.} f(x) = x^3 - 2x^2 - 4, \quad g\prs{x} = x-3\]

\begin{enumerate}
\item נכתוב את הפולינום
$f$
כולל גורמים עם מקדם
$0$.
משמאלו נכתוב את הפולינום
$g$.

\begin{otherlanguage}{english}
\[
\begin{array}{l}
{}\\
x-3\overline{) x^3 - 2x^2 + 0x - 4}
\end{array}
\]
\end{otherlanguage}

\item

נחלק את הגורם המוביל של
$f$
בזה של
$g$
ונכתוב את התוצאה למעלה.

\begin{otherlanguage}{english}
\[
\begin{array}{l}
{\hphantom{x-3)x^3 - 2}}x^2\\
x-3\overline{) x^3 - 2x^2 + 0x - 4}
\end{array}
\]
\end{otherlanguage}


\item נכפול את הגורם שהוספנו למעלה ב־%
$g$
ונכתוב את מינוס התוצאה מתחת ל־%
$f$.

\begin{otherlanguage}{english}
\[
\begin{array}{l}
{\hphantom{x-3) x^3 - 2}}x^2\\
x-3\overline{) x^3 - 2x^2 + 0x - 4} \\
\hphantom{x-3} \hspace{-8pt} -x^3 + 3x^2
\end{array}
\]
\end{otherlanguage}

\item
נסכום את השורה שכתבנו עם
$f$
ונכתוב את התוצאה מתחתיה.
``נוריד"
גורם נוסף מ־%
$f$
למטה ונוסיף אותו לתוצאה.

\begin{otherlanguage}{english}
\[
\begin{array}{l}
{\hphantom{x-3) x^3 - 2}}x^2\\
x-3\overline{) x^3 - 2x^2 + 0x - 4} \\
\hphantom{x-3} \hspace{-8pt} -x^3 + 3x^2 \\
\hphantom{x-3)} \overline{\hphantom{x^3 -3 } x^2 + 0x \hphantom{-4..}}
\end{array}
\]
\end{otherlanguage}

\item
נחזור על שלבים 2 עד 4, רק שהפעם נחלק את ההפרש שקיבלנו ב־%
$g$.

\begin{otherlanguage}{english}
\[
\begin{array}{l}
{\hphantom{x-3) x^3 - 2}}x^2 + \hphantom{0} x\\
x-3\overline{) x^3 - 2x^2 + 0x - 4} \\
\hphantom{x-3} \hspace{-8pt} -x^3 + 3x^2 \\
\hphantom{x-3)} \overline{\hphantom{x^3 -3 } x^2 + 0x \hphantom{-4..}}
\\
\hphantom{x-3)} \hphantom{x^3 -3 } \hspace{-12pt} -x^2 + 3x
\\
\hphantom{x-3 ) x^3 + 2}\overline{\hphantom{x^2 +. \hspace{-3pt}} 3x - 4 \hphantom{.}}
\end{array}
\]
\end{otherlanguage}

\item
נחזור על שלב 5 עד לקבל הפרש מדרגה שקטנה מזאת של
$g$.
ההפרש יהיה השארית
$r(x)$
והפולינום מעל
$f$
יהיה המנה
$h\prs{x}$
של
$f\prs{x}$
ב־%
$g\prs{x}$
עם שארית
$r\prs{x}$.

\begin{otherlanguage}{english}
\[
\begin{array}{l}
{\hphantom{x-3) x^3 - 2}}x^2 + \hphantom{0} x\\
x-3\overline{) x^3 - 2x^2 + 0x - 4} \\
\hphantom{x-3} \hspace{-8pt} -x^3 + 3x^2 \\
\hphantom{x-3)} \overline{\hphantom{x^3 -3 } x^2 + 0x \hphantom{-4..}}
\\
\hphantom{x-3)} \hphantom{x^3 -3 } \hspace{-12pt} -x^2 + 3x
\\
\hphantom{x-3 ) x^3 + 2}\overline{\hphantom{x^2 +. \hspace{-3pt}} 3x - 4 \hphantom{.}}
\\
\hphantom{x-3 ) x^3 + 2}{\hphantom{x^2 +. \hspace{-3pt}} \hspace{-12pt} -3x + 9 \hphantom{.}}
\\
\hphantom{x-3 ) x^3 - 2x^2 + .} \overline{\hphantom{3x -} 5 \hphantom{.}}
\end{array}
\]
\end{otherlanguage}


\end{enumerate}
\end{algorithm}

\subsection*{שורשים של פולינומים}

\begin{definition}[שורש של פולינום]
יהי
$f \in \mathbb{F}\brs{x}$.
איבר
$\alpha \in \mathbb{F}$
נקרא
\emph{
שורש של
$f$
}
אם
$f\prs{\alpha} = 0$.
\end{definition}

\begin{example}
השורשים של הפולינום
$f\prs{x} = x^2 - 1$
הם
$\set{-1,1}$.
\end{example}

\begin{theorem}[המשפט היסודי של האלגברה]
לכל
פולינום מרוכב
$f \in \mathbb{C}\brs{x}$
ממעלה לפחות
$1$
קיים שורש.
\end{theorem}

\begin{corollary}
יהי
$f \in \mathbb{C}\brs{x}$
מדרגה
$n$.
קיימים ל־%
$f$
בדיוק
$n$
שורשים
$\prs{\alpha_1, \ldots, \alpha_n}$
ב־%
$\mathbb{C}$
ומתקיים
\[f = a_n \prod_{i = 1}^n \prs{x - \alpha_n} \coloneqq a_n \prs{x - \alpha_1} \cdot \ldots \cdot \prs{x - \alpha_n}\]
כאשר
$a_n$
המקדם המוביל של
$f$.
\end{corollary}

\begin{corollary}
יהי
$f \in \mathbb{C}\brs{x}$.
אם
$\alpha \in \mathbb{C}$
שורש של
$f$,
מתקיים
\[\text{.}\prs{x - \alpha} \mid f\prs{x}\]
\end{corollary}

\begin{definition}[הריבוי של שורש]
יהי
$f \in \mathbb{C}\brs{x}$
ויהי
$\alpha \in \mathbb{C}$
שורש של
$f$.
המספר
$n$
המקסימלי עבורו
$\prs{x-\alpha}^n \mid f$
נקרא
\emph{
הריבוי של
$\alpha$
כשורש של
$f$
}
ומסומן
$m\prs{\alpha}$.
\end{definition}

\begin{definition}[שורש פשוט]
יהי
$f \in \mbb{F}\brs{x}$.
שורש
$\alpha$
של
$f$
יקרא
\emph{פשוט}
אם
$m\prs{\alpha} = 1$.
\end{definition}

\begin{proposition}
יהי
$f \in \mbb{F}\brs{x}$
ויהי
$\alpha \in \mbb{F}$.
מתקיים
$m\prs{\alpha} = \ell$
אם ורק אם
\[\text{.} f\prs{\alpha} = f'\prs{\alpha} = \ldots = f^{\prs{\ell-1}}\prs{\alpha} = 0, \quad f^{\prs{\ell}}\prs{\alpha} \neq 0\]
\end{proposition}

\begin{corollary}
שורש
$\alpha$
של פולינום
$f$
הוא פשוט אם ורק אם
$f\prs{\alpha} = 0$
וגם
$f'\prs{\alpha} \neq 0$.
\end{corollary}

\begin{theorem}[משפט השורש הרציונלי]
עבור
\[f\prs{x} = \sum_{i=0}^n a_i x^i \in \mbb{Z}\brs{x}\]
פולינום עם מקדמים שלמים,
כל שורש רציונלי של
$f$
הוא מהצורה
$\frac{p}{q}$
עבור
$p,q \in \mbb{Z}$
שלמים המקיימים
$p \mid a_0$
וגם
$q \mid a_n$.
\end{theorem}

\begin{exercise}
השתמשו במשפט השורש הרציונלי כדי למצוא את השורשים מעל
$\mbb{C}$
של הפולינום
$f\prs{x} = 2x^3 - x^2 - 4x + 2$
ואת הריבויים שלהם.
\end{exercise}

\begin{solution}
מתקיים
$f \in \mbb{Z}\brs{x}$
ולכן אפשר להשתמש במשפט השורש הרציונלי. אצלנו
$a_n = a_0 = 2$
לכן שורש רציונלי יהיה מהצורה
$\frac{p}{q}$
עבור
$p,q \in \set{\pm 1, \pm 2}$.
לכן שורש רציונלי יהיה בקבוצה
$\set{\pm \frac{1}{2}, \pm 1, \pm 2}$.
בדיקה של השורשים האפשריים תיתן שהשורש הרציונלי היחיד של
$f$
הוא
$\frac{1}{2}$.
לכן,
$x - \frac{1}{2} \mid f$
ונוכל למצוא את שאר שורשי
$f$
על ידי הסתכלות על שורשי
$g\prs{x} \ceq \frac{f}{x-\frac{1}{2}}$.

נמצא את
$\frac{f}{x-\frac{1}{2}}$
באמצעות חילוק ארוך:

\begin{otherlanguage}{english}
\[
\polylongdiv{2x^3 - x^2 - 4x + 2}{x-\frac{1}{2}}
\]
\end{otherlanguage}

קיבלנו
\[g\prs{x} = 2x^2 - 4 = \prs{\sqrt{2}x + 2}\prs{\sqrt{2}x - 2} = 2 \prs{x + \sqrt{2}}\prs{x-\sqrt{2}}\]
לכן
\[f\prs{x} = 2 \prs{x-\frac{1}{2}}\prs{x-\sqrt{2}}\prs{x+\sqrt{2}}\]
ושורשי
$f$
הם
$\set{\frac{1}{2}, \sqrt{2}, -\sqrt{2}}$,
כולם מריבוי
$1$.
\end{solution}

\begin{exercise}
יהי
\[\text{.} f\prs{x} = x^4 - 2x^3 - x^2 + 4x - 2\]
מצאו את השורשים של
$f$
ואת הריבויים שלהם.
\end{exercise}

\begin{solution}
ממשפט השורש הרציונלי, השורשים הרציונליים האפשריים של
$f$
הם
$\set{\pm 1, \pm 2}$.
מהצבה נקבל שהשורש הרציונלי היחיד של
$f$
הוא
$1$.
אם נחלק את
$f$
ב־%
$\prs{x-1}$
נקבל פולינום ממעלה
$3$
ולא נוכל להשתמש בנוסחת השורשים. לכן נבדוק קודם מה הריבוי של
$1$.
מתקיים
\begin{align*}
f'\prs{x} &= 4x^3 -6x^2 -2x + 4 \\
f''\prs{x} &= 12x^2 - 12x - 2
\end{align*}
לכן
\begin{align*}
f'\prs{1} &= 4-6-2 + 4 = 0 \\
f''\prs{1} = 12 - 12 - 2 = -2 \neq 0
\end{align*}
ולכן
$m\prs{1} = 2$.
לכן
$\prs{x-1}^2 \mid f\prs{x}$.
נחשב את המנה
\[g\prs{x} \ceq \frac{f\prs{x}}{\prs{x-1}^2} = \frac{f\prs{x}}{x^2 - 2x + 1}\]
ונקבל
\begin{otherlanguage}{english}
\[
\polylongdiv{x^4 - 2x^3 - x^2 + 4x - 2}{x^2 - 2x + 1}
\]
\end{otherlanguage}

כלומר
\[f\prs{x} = \prs{x-1}^2 \prs{x^2 - 2} = \prs{x-1}^2 \prs{x+\sqrt{2}}\prs{x-\sqrt{2}}\]
לכן שורשי
$f$
הם
$1$
מריבוי
$2$
ו־%
$\set{\pm \sqrt{2}}$
כל אחד מריבוי
$1$.
\end{solution}

\subsection*{נוסחאות וייטה}

לפעמים נרצה תיאור נוח לסכומים או מכפלות של שורשי פולינום.
אם נתון פולינום
\[f\prs{x} = \sum_{i = 0}^n a_i x^i\]
מדרגה
$n$
ועם שורשים
$\prs{r_i}_{i = 1}^n$,
נכתוב
\[f\prs{x} = a_n \prod_{i = 1}^n \prs{x-r_i}\]
ונקבל כי המקדם החופשי הוא
\[a_0 = a_n \prs{-1}^n \prod_{i=1}^n r_i\]
לכן
\[\text{.} \prod_{i=1}^n r_i = \prs{-1}^n \frac{a_0}{a_n} \]
נקבל גם
\[a_{n-1} = a_n\prs{-r_1 - \ldots - r_n}\]
ולכן
\[\text{.} \sum_{i=1}^n r_i = -\frac{a_{n-1}}{a_n}\]

נוסחאות וייטה מתארות קשר כללי יותר בין מקדמי פולינום לשורשיו.

\begin{theorem}[נוסחאות וייטה]
יהי
\[f\prs{x} = \sum_{i = 0}^n a_i x^i\]
פולינום מדרגה
$n$
מעל
$\mbb{R}$
או
$\mbb{C}$.

יהיו
$\prs{r_i}_{i = 1}^n$
שורשי
$f$.
לכל
$k \in \set{1, \ldots, n}$
מתקיים
\[\text{.} \sum_{1 \leq i_1 < \ldots < i_k \leq n} \prs{\prod_{j=1}^k r_{i_j}} = \prs{-1}^k \frac{a_{n-k}}{a_n}\]
\end{theorem}

\begin{exercise}
מצאו את שורשי הפולינום
\[\text{.} f\prs{x} = x^3 - \prs{2 + \sqrt{2}}x^2 + \prs{2 \sqrt{2} + 1}x - \sqrt{2}\]
\end{exercise}

\begin{solution}
נשים לב שסכום מקדמי הפולינום שווה
$0$,
לכן
$1$
שורש של
$f$.
מתקיים
\begin{align*}
f'\prs{x} = 3x^2 - \prs{4 + 2\sqrt{2}} x + 2 \sqrt{2} + 1 \\
f''\prs{x} = 6x - 4 - 2 \sqrt{2}
\end{align*}
לכן
\begin{align*}
f'\prs{1} &= 3 - 4 - 2\sqrt{2} + 2 \sqrt{2} + 1 = \sqrt{2} = 0 \\
f''\prs{1} &= 6-4-2 = 0
\end{align*}
ולכן
$m\prs{1} = 2$.
מנוסחאת וייטה נקבל שסכום שורשי
$f$
שווה
$2+\sqrt{2}$
לכן השורש הנוסף הוא
$\sqrt{2}$
מריבוי
$1$.
\end{solution}

\begin{exercise}
יהי
\[f\prs{x} = -4 x^3 + a_2 x^2 + a_0\]
פולינום מעל
$\mbb{C}$
שתלוי בנעלמים
$a_0, a_2$.
נתון כי סכום שורשי
$f$
שווה
$4$
וכי מכפלתם שווה
$-3$.
מצאו את שורשי
$f$
ואת הריבויים שלהם.
\end{exercise}

\begin{solution}
נסמן את שורשי הפולינום
$\set{r_1, r_2, r_3}$.
מנוסחאת וייטה עבור
$k = 1$
נקבל
\[4 = r_1 + r_2 + r_3 = - \frac{a_2}{a_3}\]
לכן
\[a_2 = -4 a_3 = \prs{-4}^2 = 16 \text{.}\]

מנוסחת וייטה עבור
$k=3$
נקבל
\[-3 = r_1 r_2 r_3 = \prs{-1}^3 \frac{a_0}{a_n}\]
כלומר
\[\text{.} a_0 = 3 a_n = -12\]

לכן
\[\text{.} f\prs{x} = -4x^3 + 16x^2 - 12\]
ל־%
$f$
אותם שורשים כמו ל־%
$-\frac{f}{4}$
כיוון שכפל בקבוע שונה מאפס אינו משפיע על שורשי פולינום.
לכן נחפש שורשים עבור
\[\text{.} g\prs{x} \coloneqq -\frac{f\prs{x}}{4} = x^3 - 4x^2 + 3\]
מקדמי
$g$
שלמים ולכן נוכל להיעזר במשפט השורש הרציונלי. נקבל שהשורשים הרציונליים של
$g$
הם מהצורה
$\frac{p}{q}$
עבור
$p,q \in \mbb{Z}$
המקיימים
$p \mid 1$
ו־%
$q \mid 3$.
לכן, השורשים הרציונליים של
$g$
בקבוצה
$\set{\pm 1, \pm \frac{1}{3}}$.
נשים לב כי
$1$
אכן שורש של
$g$.
נבצע חילוק פולינומים

\begin{otherlanguage}{english}
\[
\polylongdiv{x^3 - 4x^2 + 3}{x-1}
\]
\end{otherlanguage}

ונקבל כי
\[\text{.} g\prs{x} = \prs{x-1}\prs{x^2 - 3x - 3}\]
את שורשי
$x^2 - 3x - 3$
נמצא בעזרת נוסחת השורשים ונקבל
\[\text{.} g\prs{x} = \prs{x-1} \prs{x-\frac{3}{2} - \frac{\sqrt{21}}{2}} \prs{x - \frac{3}{2} + \frac{\sqrt{21}}{2}}\]
לכן שורשי
$f$
הם
$\set{1, \frac{3}{2} \pm \frac{\sqrt{21}}{2}}$,
כל אחד מהם מריבוי
$1$.
\end{solution}

\end{document}